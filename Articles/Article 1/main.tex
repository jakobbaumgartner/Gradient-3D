\documentclass[]{article}

 \usepackage{easyReview}


%opening
\title{\comment{3D Voxel Grid Based Path Planning for Robotic Manipulators using Matrix Multiplication Technique}{Alternatives:\\ - Efficient distance calculation / Efficient repulsive field calculation technique \\   - Matrix Multiplication-Driven Repulsive Fields for 3D Voxel-Based Robotic Manipulator Path Planning \\ - Robotic Manipulator Path Planning Optimization Using Matrix-Derived Repulsive Fields Based on 3D Voxel Grid}}

\author{Jakob Baumgartner, Gregor Klančar}

\begin{document}

\maketitle

\begin{abstract}

\end{abstract}

\section{Introduction}

\section{Background}

\todo{1-2 PAGES}

\add{
	- PRESENT DISTANCE CALCULATION FOR MANIPULATORS (sensors, lidar, ir, bounding boxes) \\
	- PRESENT PATH PLANNING METHODS FOR MANIPULATOR (optimization, sampling, biological, learning) \\
	- similar to Distance Transform (a kind of inverted distance transform) \\
	- method was inspired by Khatib APF (however, it evolved into a different method) \\
	- different existing APF manipulator implementation articles \\
	- VFH \\
}

\section{Methodology}

\todo{3 PAGES}

\subsection{Optimization Algorithm}

\add{
	- optimization algorithm \\
	- robot kinematics (include the exact-reduced method) \\
	- primary task of distance goal \\
	- secondary task of repulsive field \\
	- damped least squares \\
	- task slowdown option \\
	- secondary task of manipulation measure \\	
}



\subsection{Task Constraints}

\add{-equations of occupied and empty space}

\subsection{Attractive Velocity}

\add{	
	- attractive field calculation (normalization of attractive force) 
}

\subsection{Repulsive Velocity}

\add{	
	- object detection is done in task domain and not c-space (more logical) \\
	- repulsive field calculation - matrix "convolution" method \\
	- matrix size and shape selection \\
	- equation for repulsive kernel values (non-linear) \\
	- PLOT: (ERK) kernel graphics \\
	- PLOT: (ERK) linear kernel graphics \\
	- PLOT: kernel field shape 	\\
	- interpolation of the repulsive field \\
	- what if there are obstacles behind wall (usually not the case, depth sensors show only thin walls, some noise doesnt matter, non-linear kernel, possible additional pre-convolution to convert obstacle grid to edges) \\
	- effecient calculation in dinamic environments, lacking prediction capabilities (MPC) \\
}

\subsubsection{Kernel Selection}

\subsubsection{Interpolation}



\section{Implementation}

\section{Results}

\todo{3 PAGES}

\add{
	- include execution times \\
	- PLOT: kernel on robot graphics \\
}

\section{Discussion}

\todo{1 PAGE}

\add{
	- add the limitations of such method (already mentioned by Khatib) \\
	- the limitations of local search \\
	- number of parameters that need to be tuned (are there actually that many?)
}

\section{Conclusion}



\end{document}
