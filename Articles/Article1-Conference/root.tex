%%%%%%%%%%%%%%%%%%%%%%%%%%%%%%%%%%%%%%%%%%%%%%%%%%%%%%%%%%%%%%%%%%%%%%%%%%%%%%%%
%2345678901234567890123456789012345678901234567890123456789012345678901234567890
%        1         2         3         4         5         6         7         8

\documentclass[letterpaper, 10 pt, conference]{ieeeconf}  % Comment this line out if you need a4paper



%\documentclass[a4paper, 10pt, conference]{ieeeconf}      % Use this line for a4 paper

\IEEEoverridecommandlockouts                              % This command is only needed if 
                                                          % you want to use the \thanks command

\overrideIEEEmargins                                      % Needed to meet printer requirements.

%In case you encounter the following error:
%Error 1010 The PDF file may be corrupt (unable to open PDF file) OR
%Error 1000 An error occurred while parsing a contents stream. Unable to analyze the PDF file.
%This is a known problem with pdfLaTeX conversion filter. The file cannot be opened with acrobat reader
%Please use one of the alternatives below to circumvent this error by uncommenting one or the other
%\pdfobjcompresslevel=0
%\pdfminorversion=4

% See the \addtolength command later in the file to balance the column lengths
% on the last page of the document

% The following packages can be found on http:\\www.ctan.org
%\usepackage{graphics} % for pdf, bitmapped graphics files
%\usepackage{epsfig} % for postscript graphics files
%\usepackage{mathptmx} % assumes new font selection scheme installed
%\usepackage{times} % assumes new font selection scheme installed
%\usepackage{amsmath} % assumes amsmath package installed
%\usepackage{amssymb}  % assumes amsmath package installed
\usepackage{hyperref}


 \usepackage{easyReview}
 


\title{\LARGE \bf
Preparation of Papers for IEEE Sponsored Conferences \& Symposia*
}


\author{Albert Author$^{1}$ and Bernard D. Researcher$^{2}$% <-this % stops a space
\thanks{*This work was not supported by any organization}% <-this % stops a space
\thanks{$^{1}$Albert Author is with Faculty of Electrical Engineering, Mathematics and Computer Science,
        University of Twente, 7500 AE Enschede, The Netherlands
        {\tt\small albert.author@papercept.net}}%
\thanks{$^{2}$Bernard D. Researcheris with the Department of Electrical Engineering, Wright State University,
        Dayton, OH 45435, USA
        {\tt\small b.d.researcher@ieee.org}}%
}


\begin{document}



\maketitle
\thispagestyle{empty}
\pagestyle{empty}


%%%%%%%%%%%%%%%%%%%%%%%%%%%%%%%%%%%%%%%%%%%%%%%%%%%%%%%%%%%%%%%%%%%%%%%%%%%%%%%%
\begin{abstract}

This electronic document is a Olive template. The various components of your paper [title, text, heads, etc.] are already defined on the style sheet, as illustrated by the portions given in this document.

\end{abstract}


%%%%%%%%%%%%%%%%%%%%%%%%%%%%%%%%%%%%%%%%%%%%%%%%%%%%%%%%%%%%%%%%%%%%%%%%%%%%%%%%
\section{INTRODUCTION}

% PREDSTAVITEV REDUNDANCE

Kinematic redundancy enables a manipulator to follow a predefined task space trajectory using the endeffector (EE), while simultaneously, optimising for an additional task with the remaining movement capacity without impacting the trajectory adherence. This is possible because the robot's degrees of freedom (DOF) go beyond what is required to perform the primary task. Consequently, the robot can adopt different joint configurations optimised according to the secondary task while performing the primary task. Common secondary tasks include avoiding singularities, optimising the manipulability measure, minimising joint torques and avoiding obstacles in the operating space.

\add{citations}

% PREDSTAVITEV PRISTOPOV PLANIRANJA POTI V ROBOTIKI
The task of finding the joint trajectories of a manipulator is called motion planning~\cite{IDEASLab2023}. It consists of finding a sequence of joint configurations for a robot so that the robot can move along this path from its initial configuration to the goal configuration without colliding with itself, static obstacles or other agents in the environment. In addition to collision avoidance, motion planning for manipulators can optionally take into account various constraints, such as position, velocity, acceleration or jerk constraints for the joint angle or end effector, precision of the end effector with respect to position and orientation, stability of the manipulator, avoidance of singularities, or any number of other criteria.

There are numerous methods for planning manipulator movements, they can rougly be seperated into global and local approaches. 

Sampling-based methods such as PRM, RRT, RRT*, RRT-Connect, Informerd RRT*, BIT* and others offer a solution based on a global search in configuration space. However, the generated trajectories are not always smooth or optimal, and the performance of the methods may be insufficient for real-time operation. 

Recently, a number of learning-based methods using data-driven techniques have been proposed to improve or accelerate the functionality of sampling-based methods. \add{examples} 

Trajectory optimisation methods such as CHOMP, STOMP and TrajOpt, on the other hand, use optimization to improve an initial seed trajectory. Consequently, the optimality of the solution is highly dependent on this initial trajectory. Nevertheless, these methods are capable of generating smooth trajectories, and although they can be too computationally intensive for high DOF dynamic real-time environments, they are generally effective in finding constrained motion plans. 

Local motion planning approaches employ optimization techniques, two common ones are inverse kinematics~\cite{c29,c38}, that is based on finding a least squares solution of the manipulator joint velocities, and quadratic programming (QP)~\cite{c21,c22,c23}. In the following text we will focus on inverse kinematics based approaches.

\remove{Different control schemes have been proposed. Acceleration-based control excels in precise handling of motion changes, velocity-based control offers consistent and smooth movement, while force and torque-based control provides direct control over joint forces for robust physical interactions. (CITATIONS in COMMENT BELLOW)}
%\remove{Different control schemes have been proposed~\cite{c24}. Acceleration-based control excels in precise handling of motion changes, velocity-based control offers consistent and smooth movement, while force and torque-based control~\cite{c25, c26, c27, c28, c31} provides direct control over joint forces for robust physical interactions.}

% ----

%In our work we use velocity control strategy for obstacle avoidance using the manipulator. naša primarna naloga premik vrha robota - end efektorja (EE) v želeno pozo, to je točko in orientacijo v prostoru naloge. Naša sekundarna naloga se tako izvaja v preostalem prostoru gibanja sklepov redundance robota (DOF, DOR). ??? je predlagal razširitev jakobijeve matrike end-efektorja z dodatno matriko transformacije izogibne hitrosti iz kartzezičnega prostora v prostor sklepov. Če nalogi nista popolnoma neodvisni ena od druge, in njune sklepne hitrosti niso medsebojno ortagonalne, lahko to pripelje do medsebojnega oviranja izbajanja nalog. Zato je ??? predlagal uporabo transformacijo hitrosti sekundarne naloge v  ničelnega prostora hitrosti primarne nalgoe, to je v prostor ortagonalnih hitrosti. Posledično izvajanje sekundarne naloge ne vpliva neposredno v izvajanje primarne naloge, gledano le v enem trenutku izvajanja. 
%
%Različni razisklovalci so se lotili problema kinematičnega izogibanja na različne načine, kot je opisano v \cite{c30, c39}. 
%
%Colbaugh in Glass  1989 \cite{c31, c32} sta se problema lotila v dveh korakih. V prvem koraku sta izračunala trajektorijo vrha robota, v drugem pa sta uporabila optimizacijo za izboljšanje dinamičnega odziva robota med izvajanjem trajektorije in za druge naloge, kot so izogibanje ovir.  
%
%Sciavicco and Siciliano \cite{c34, c35} and Egeland~\cite{c36} independetly introduced the concept ot task-space augmentation. It was later revisited by searji~\cite{c37}. The Solution is to stack the remaining rows in Jacobian with secondary task Jacobian into square matrix, that returns only one solution.
%
%Macijewski in Klein (Maciejewski & Klein, 1985) \cite{c29}, Nakamura in Hanafusa 1987 \cite{c38} predstavijo koncept prioritete nalog in ničelnega prostora za izogibanje, pri čemer se sekundarna naloga lahko giblje le s hitrostmi, ki ne vplivajo neposredno na gibanje primarne naloge. pri čemer sta izogibne sklepne hitrosti izračunala funkcijo najmanjše razdalje med točko na manipulatorju in točko izogibanja / objekta / ovire.  
%
%Omenjeni pristopi se ne osredotočajo na okolje, tega pogosto prikažejo kot geometrijske primitive in nato razdaljo izračunajo kot razdaljo med primitivom in premicami, ki opisujejo manipulator. 
%
%Khatib \cite{c33} proposes the concept of Artificial Potential Field. Polje sestavljata odbojna komponenta, ki end effektor odbija od ovir, ki so predstavljene kot geometrijski primitivi in privlačna komponenta, ki manipulator privlači proti cilju. 
%
%
%

Various researchers have adopted different approaches to kinematic avoidance, as detailed in~\cite{c30, c39}.

Colbaugh and Glass (1989)~\cite{c31, c32} tackled this problem in a two-step process. Initially, they calculated the robot's end-effector trajectory. Subsequently, they used optimization to enhance the robot's dynamic response during trajectory execution and for tasks like obstacle avoidance.

Sciavicco and Siciliano~\cite{c34, c35}, as well as Egeland~\cite{c36}, independently introduced the concept of task-space augmentation, later revisited by Searji~\cite{c37}. This method involves integrating the remaining rows of the Jacobian with the secondary task Jacobian into a square matrix, yielding a singular solution.

Maciejewski and Klein (1985)~\cite{c29}, and Nakamura and Hanafusa (1987)~\cite{c38}, presented the concept of task prioritization and a null space for avoidance. In this framework, the secondary task is restricted to velocities that do not directly affect the primary task's movement. Their method calculates avoidance joint velocities based on the minimum distance between a point on the manipulator and an avoidance point, object, or obstacle.

These approaches typically do not focus on the environmental context, often representing it as geometric primitives and then calculating the distance between these primitives and the lines describing the manipulator.

Khatib~\cite{c33} proposes the concept of an Artificial Potential Field. This field consists of a repulsive component that deflects the end effector away from obstacles, depicted as geometric primitives, and an attractive component that draws the manipulator towards its target.

\add{\\
THESE SEMINAL WORKS HAVE BEEN FURTHER DEVELOPED / APPLIED IN ... add 5-10 modern citations
\\
\\
}


\add{
 -used for obstacle avoidance \\
 -used for task manip, ... , ...
 -fix citation styles
}

 The inverse kinematics with task prioritisation,  Both methods are fast, suitable for real-time applications in dynamic environments and provide smooth solutions. However, since they do not plan further than one step ahead, they tend to get stuck in local minima. Therefore, they are often combined with a higher-level planner, for global static environment based \replace{motion}{path} planning, while local optimisation takes dynamic environment changes into account.

\add{Maybe add optimization - global intiger programming based approaches, by Tedrake.}


%The inverse kinematics with task prioritisation, which is based on finding a least squares solution of the manipulator joint velocities, and linear quadratic programming (LQR) are two of the local optimization methods. Both methods are fast, suitable for real-time applications in dynamic environments and provide smooth solutions. However, since they do not plan further than one step ahead, they tend to get stuck in local minima. Therefore, they are often combined with a higher-level planner, for global static environment based \replace{motion}{path} planning, while local optimisation takes dynamic environment changes into account.

\add{citations}

% PREDSTAVITEV PRISTOPOV RAČUNANJA ODDALJENOSTI OVIRAM
To be able to generate collision free trajectory we need to have a representation of the static environment. One common approach is to represent environment with geomertic primitives such as bounding boxes or spheres. Another way is to divide the occupied space into hierarhical trees, such as octatree, to make calculations more efficient. Such representations are memory effective and offer simple and computationally undemanding distance calculation and collision detection. These representations are widely used in computer simulations and games, however they can be suboptimal for real environment, where sensor measurements arent always accurate and contain only part of the surrounding space. This can lead to approximation errors, which are additionally probelmatic when we are dealing with irregularly shaped objects.

Point-cloud representation captures the environment by directly using a collection of points obtained from sensors like LiDAR, radar of RGBD camera, providing a more precise depiction of space than bounding boxes or spheres. This method can more accurately represent the complex and irregular shapes of real-world objects, although it typically requires more memory and computational power to process, which can make it chllenging for real-time representation. Point cloud data is often further converted into voxel grids.

Voxel grid representation divides the space into a regular grid of volumetric elements, or voxels, which can be used to create a more manageable approximation of the environment. While this approach offers a balance between detail and computational efficiency, it can still introduce discretization errors, particularly when modeling objects with smooth surfaces or intricate details. The fidelity of the representation is dependent on the size of the voxels: smaller voxels can capture more detail but require more memory and computation, while larger voxels result in coarser approximations but are more memory-efficient, adaptive voxel grids have been explored, where the voxel size can vary throughout the space to provide higher resolution in regions of interest while conserving resources in less critical areas. Voxel grids can incorporate probabilistic information, such as in occupancy grid maps, where each voxel holds a probability indicating the chance of an obstacle's presence. The occupancy probability of a voxel can be updated dynamically using sensor measurements and Bayesian updating methods. As new sensor data is collected, the probabilities are revised to reflect the increased or decreased likelihood of the presence of an obstacle in the voxel space. Additionally, techniques such as ESDF (Euclidian Signed Distance Field) and TSDF (Truncated Signed Distance Field) can be utilized in conjunction with voxel grids to encode signed distance information, that describes what is the distance between the selected voxel field and the nearest occupied field.

There has been a number of proposed algorithms for calculating ESDF. One common method is the Brushfire method, that spreads from obstacles until it calculates the distance for every field on a grid. \add{distance transforms} Jump Flooding Algorithm (JFA) is a similar method, that can be implemented on a GPU for faster parallel distance calculation.  
\add{\\-Wavefront \\
	 -D* Klančar\\
 	 -Euclidian distnace algorithms\\
 	 -distance transforms\\
  	 -voronoi diagrams\\
   	 -voxblox, FIESTA \\
     -ray casting, bullet casting ...}

\clearpage

\section{USING THE TEMPLATE}

Use this sample document as your LaTeX source file to create your document. Save this file as {\bf root.tex}. You have to make sure to use the cls file that came with this distribution. If you use a different style file, you cannot expect to get required margins. Note also that when you are creating your out PDF file, the source file is only part of the equation. {\it Your \TeX\ $\rightarrow$ PDF filter determines the output file size. Even if you make all the specifications to output a letter file in the source - if your filter is set to produce A4, you will only get A4 output. }

It is impossible to account for all possible situation, one would encounter using \TeX. If you are using multiple \TeX\ files you must make sure that the ``MAIN`` source file is called root.tex - this is particularly important if your conference is using PaperPlaza's built in \TeX\ to PDF conversion tool.

\subsection{Headings, etc}

Text heads organize the topics on a relational, hierarchical basis. For example, the paper title is the primary text head because all subsequent material relates and elaborates on this one topic. If there are two or more sub-topics, the next level head (uppercase Roman numerals) should be used and, conversely, if there are not at least two sub-topics, then no subheads should be introduced. Styles named  Heading 1 ,  Heading 2 ,  Heading 3 , and  Heading 4  are prescribed.

\subsection{Figures and Tables}

Positioning Figures and Tables: Place figures and tables at the top and bottom of columns. Avoid placing them in the middle of columns. Large figures and tables may span across both columns. Figure captions should be below the figures; table heads should appear above the tables. Insert figures and tables after they are cited in the text. Use the abbreviation  Fig. 1 , even at the beginning of a sentence.

\begin{table}[h]
\caption{An Example of a Table}
\label{table_example}
\begin{center}
\begin{tabular}{|c||c|}
\hline
One & Two\\
\hline
Three & Four\\
\hline
\end{tabular}
\end{center}
\end{table}


   \begin{figure}[thpb]
      \centering
      \framebox{\parbox{3in}{We suggest that you use a text box to insert a graphic (which is ideally a 300 dpi TIFF or EPS file, with all fonts embedded) because, in an document, this method is somewhat more stable than directly inserting a picture.
}}
      %\includegraphics[scale=1.0]{figurefile}
      \caption{Inductance of oscillation winding on amorphous
       magnetic core versus DC bias magnetic field}
      \label{figurelabel}
   \end{figure}
   

Figure Labels: Use 8 point Times New Roman for Figure labels. Use words rather than symbols or abbreviations when writing Figure axis labels to avoid confusing the reader. As an example, write the quantity  Magnetization , or  Magnetization, M , not just  M . If including units in the label, present them within parentheses. Do not label axes only with units. In the example, write  Magnetization (A/m)  or  Magnetization {A[m(1)]} , not just  A/m . Do not label axes with a ratio of quantities and units. For example, write  Temperature (K) , not  Temperature/K. 

\section{CONCLUSIONS}

A conclusion section is not required. Although a conclusion may review the main points of the paper, do not replicate the abstract as the conclusion. A conclusion might elaborate on the importance of the work or suggest applications and extensions. 

\addtolength{\textheight}{-12cm}   % This command serves to balance the column lengths
                                  % on the last page of the document manually. It shortens
                                  % the textheight of the last page by a suitable amount.
                                  % This command does not take effect until the next page
                                  % so it should come on the page before the last. Make
                                  % sure that you do not shorten the textheight too much.

%%%%%%%%%%%%%%%%%%%%%%%%%%%%%%%%%%%%%%%%%%%%%%%%%%%%%%%%%%%%%%%%%%%%%%%%%%%%%%%%



%%%%%%%%%%%%%%%%%%%%%%%%%%%%%%%%%%%%%%%%%%%%%%%%%%%%%%%%%%%%%%%%%%%%%%%%%%%%%%%%



%%%%%%%%%%%%%%%%%%%%%%%%%%%%%%%%%%%%%%%%%%%%%%%%%%%%%%%%%%%%%%%%%%%%%%%%%%%%%%%%
\section*{APPENDIX}

Appendixes should appear before the acknowledgment.

\section*{ACKNOWLEDGMENT}

The preferred spelling of the word  acknowledgment  in America is without an  e  after the  g . Avoid the stilted expression,  One of us (R. B. G.) thanks . . .   Instead, try  R. B. G. thanks . Put sponsor acknowledgments in the unnumbered footnote on the first page.



%%%%%%%%%%%%%%%%%%%%%%%%%%%%%%%%%%%%%%%%%%%%%%%%%%%%%%%%%%%%%%%%%%%%%%%%%%%%%%%%

References are important to the reader; therefore, each citation must be complete and correct. If at all possible, references should be commonly available publications.



\begin{thebibliography}{99}
	
\bibitem{IDEASLab2023}
IDEAS Lab, "Motion and Path Planning," presented at Purdue University, 2023. [Online]. Available: \url{https://ideas.cs.purdue.edu/research/robotics/planning/}. Accessed on: Jan. 9, 2024.
	
\bibitem{c21} E. A. Basso and K. Y. Pettersen, “Task-Priority Control of Redundant Robotic Systems using Control Lyapunov and Control Barrier Function based Quadratic Programs,” in IFAC-PapersOnLine, vol. 53, no. 2, pp. 9037–9044, 2020. doi: 10.1016/j.ifacol.2020.12.2024.


\bibitem{c22} H. Toshani and M. Farrokhi, “Real-time inverse kinematics of redundant manipulators using neural networks and quadratic programming: A Lyapunov-based approach,” Robotics and Autonomous Systems, vol. 62, no. 6, pp. 766–781, Jun. 2014. doi: 10.1016/j.robot.2014.02.005.

\bibitem{c23} Y. Zhang, S. S. Ge, and T. H. Lee, “A Unified Quadratic-Programming-Based Dynamical System Approach to Joint Torque Optimization of Physically Constrained Redundant Manipulators,” IEEE Trans. Syst., Man, Cybern. B, vol. 34, no. 5, pp. 2126–2132, Oct. 2004. doi: 10.1109/TSMCB.2004.830347.

\bibitem{c24} J. Nakanishi, R. Cory, M. Mistry, J. Peters, and S. Schaal, “Comparative experiments on task space control with redundancy resolution,” in Proc. 2005 IEEE/RSJ Int. Conf. on Intelligent Robots and Systems, Edmonton, Alta., Canada, 2005, pp. 3901–3908. doi: 10.1109/IROS.2005.1545203.

\bibitem{c25} M. H. Raibert and J. J. Craig, “Hybrid Position/Force Control of Manipulators,” Journal of Dynamic Systems, Measurement, and Control, vol. 103, no. 2, pp. 126–133, Jun. 1981. doi: 10.1115/1.3139652.

\bibitem{c26} T. Yoshikawa, “Dynamic hybrid position/force control of robot manipulators--Description of hand constraints and calculation of joint driving force,” IEEE J. Robot. Automat., vol. 3, no. 5, pp. 386–392, Oct. 1987. doi: 10.1109/JRA.1987.1087120.

\bibitem{c27} O. Khatib, “A unified approach for motion and force control of robot manipulators: The operational space formulation,” IEEE J. Robot. Automat., vol. 3, no. 1, pp. 43–53, Feb. 1987. doi: 10.1109/JRA.1987.1087068.

\bibitem{c28} N. Hogan, “Impedance Control: An Approach to Manipulation,” in Proc. 1984 American Control Conf., San Diego, CA, USA, Jul. 1984, pp. 304–313. doi: 10.23919/ACC.1984.4788393.

\bibitem{c29} A. A. Maciejewski and C. A. Klein, “Obstacle Avoidance for Kinematically Redundant Manipulators in Dynamically Varying Environments,” The International Journal of Robotics Research, vol. 4, no. 3, pp. 109–117, Sep. 1985. doi: 10.1177/027836498500400308.

\bibitem{c30} L. Lajpah and T. Petri, “Obstacle Avoidance for Redundant Manipulators as Control Problem,” in Serial and Parallel Robot Manipulators - Kinematics, Dynamics, Control and Optimization, S. Kucuk, Ed. InTech, 2012. doi: 10.5772/32651.

\bibitem{c31} R. Colbaugh and K. Glass, “Cartesian control of redundant robots,” J. Robotic Syst., vol. 6, no. 4, pp. 427–459, Aug. 1989. doi: 10.1002/rob.4620060409.

\bibitem{c32} K. Glass, R. Colbaugh, D. Lim, and H. Seraji, “Real-time collision avoidance for redundant manipulators,” IEEE Trans. Robot. Automat., vol. 11, no. 3, pp. 448–457, Jun. 1995. doi: 10.1109/70.388789.

\bibitem{c33} O. Khatib, “Real-time obstacle avoidance for manipulators and mobile robots,” in 1985 IEEE International Conference on Robotics and Automation Proceedings, Mar. 1985, pp. 500–505. doi: 10.1109/ROBOT.1985.1087247.

\bibitem{c34} L. Sciavicco and B. Siciliano, “A solution algorithm to the inverse kinematic problem for redundant manipulators,” IEEE J. Robot. Automat., vol. 4, no. 4, pp. 403–410, Aug. 1988. doi: 10.1109/56.804.

\bibitem{c35} L. Sciavicco and B. Siciliano, “Solving the Inverse Kinematic Problem for Robotic Manipulators,” in RoManSy 6, A. Morecki, G. Bianchi, and K. Kedzior, Eds., Boston, MA: Springer US, 1987, pp. 107–114. doi: 10.1007/978-1-4684-6915-8\_9.

\bibitem{c36} O. Egeland, “Task-space tracking with redundant manipulators,” IEEE J. Robot. Automat., vol. 3, no. 5, pp. 471–475, Oct. 1987. doi: 10.1109/JRA.1987.1087118.

\bibitem{c37} H. Seraji, “Configuration control of redundant manipulators: theory and implementation,” IEEE Trans. Robot. Automat., vol. 5, no. 4, pp. 472–490, Aug. 1989. doi: 10.1109/70.88062.

\bibitem{c38} Y. Nakamura, H. Hanafusa, and T. Yoshikawa, “Task-Priority Based Redundancy Control of Robot Manipulators,” The International Journal of Robotics Research, vol. 6, no. 2, pp. 3–15, Jun. 1987. doi: 10.1177/027836498700600201.

\bibitem{c39} B. Siciliano and O. Khatib, Eds., Springer Handbook of Robotics. in Springer Handbooks. Cham: Springer International Publishing, 2016. doi: 10.1007/978-3-319-32552-1.


\end{thebibliography}




\end{document}
