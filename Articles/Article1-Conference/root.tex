%%%%%%%%%%%%%%%%%%%%%%%%%%%%%%%%%%%%%%%%%%%%%%%%%%%%%%%%%%%%%%%%%%%%%%%%%%%%%%%%
%2345678901234567890123456789012345678901234567890123456789012345678901234567890
%        1         2         3         4         5         6         7         8

\documentclass[letterpaper, 10 pt, conference]{ieeeconf}  % Comment this line out if you need a4paper



%\documentclass[a4paper, 10pt, conference]{ieeeconf}      % Use this line for a4 paper

\IEEEoverridecommandlockouts                              % This command is only needed if 
                                                          % you want to use the \thanks command

\overrideIEEEmargins                                      % Needed to meet printer requirements.

%In case you encounter the following error:
%Error 1010 The PDF file may be corrupt (unable to open PDF file) OR
%Error 1000 An error occurred while parsing a contents stream. Unable to analyze the PDF file.
%This is a known problem with pdfLaTeX conversion filter. The file cannot be opened with acrobat reader
%Please use one of the alternatives below to circumvent this error by uncommenting one or the other
%\pdfobjcompresslevel=0
%\pdfminorversion=4

% See the \addtolength command later in the file to balance the column lengths
% on the last page of the document

% The following packages can be found on http:\\www.ctan.org
%\usepackage{graphics} % for pdf, bitmapped graphics files
%\usepackage{epsfig} % for postscript graphics files
%\usepackage{mathptmx} % assumes new font selection scheme installed
%\usepackage{times} % assumes new font selection scheme installed
%\usepackage{amsmath} % assumes amsmath package installed
%\usepackage{amssymb}  % assumes amsmath package installed
\usepackage{hyperref}


 \usepackage{easyReview}
 


\title{\LARGE \bf
Preparation of Papers for IEEE Sponsored Conferences \& Symposia*
}


\author{Albert Author$^{1}$ and Bernard D. Researcher$^{2}$% <-this % stops a space
\thanks{*This work was not supported by any organization}% <-this % stops a space
\thanks{$^{1}$Albert Author is with Faculty of Electrical Engineering, Mathematics and Computer Science,
        University of Twente, 7500 AE Enschede, The Netherlands
        {\tt\small albert.author@papercept.net}}%
\thanks{$^{2}$Bernard D. Researcheris with the Department of Electrical Engineering, Wright State University,
        Dayton, OH 45435, USA
        {\tt\small b.d.researcher@ieee.org}}%
}


\begin{document}



\maketitle
\thispagestyle{empty}
\pagestyle{empty}


%%%%%%%%%%%%%%%%%%%%%%%%%%%%%%%%%%%%%%%%%%%%%%%%%%%%%%%%%%%%%%%%%%%%%%%%%%%%%%%%
\begin{abstract}

This electronic document is a Olive template. The various components of your paper [title, text, heads, etc.] are already defined on the style sheet, as illustrated by the portions given in this document.

\end{abstract}


%%%%%%%%%%%%%%%%%%%%%%%%%%%%%%%%%%%%%%%%%%%%%%%%%%%%%%%%%%%%%%%%%%%%%%%%%%%%%%%%
\section{INTRODUCTION}

% PREDSTAVITEV REDUNDANCE

Kinematic redundancy enables a manipulator to follow a predefined task space trajectory using the endeffector (EE), while simultaneously, optimising for an additional task with the remaining movement capacity without impacting the trajectory adherence. This is possible because the robot's degrees of freedom (DOF) go beyond what is required to perform the primary task. Consequently, the robot can adopt different joint configurations optimised according to the secondary task while performing the primary task. Common secondary tasks include avoiding singularities, optimising the manipulability measure, minimising joint torques and avoiding obstacles in the operating space.

\add{citations}

% PREDSTAVITEV PRISTOPOV PLANIRANJA POTI V ROBOTIKI

Motion planning~\cite{IDEASLab2023} is a fundamental problem in robotics. It consists of finding a sequence of joint configurations for a robot so that the robot can move along this path from its initial configuration to the goal configuration without colliding with static obstacles or other robots in the environment. In addition to collision avoidance, motion planning for manipulators can optionally take into account various constraints, such as position, velocity, acceleration or jerk constraints for the joint angle or end effector, precision of the end effector with respect to position and orientation, stability of the manipulator, avoidance of singularities, or any number of other criteria.

There are numerous methods for planning manipulator movements. Sampling-based methods such as PRM, RRT, RRT*, RRT-Connect, Informerd RRT*, BIT* and others offer a solution based on a global search in configuration space. However, the generated trajectories are not always smooth or optimal, and the performance of the methods may be insufficient for real-time operation. Recently, a number of learning-based methods using data-driven techniques have been proposed to improve or accelerate the functionality of sampling-based methods. \add{examples} Trajectory optimisation methods such as CHOMP, STOMP and TrajOpt, on the other hand, use optimization to improve an initial seed trajectory. Consequently, the optimality of the solution is highly dependent on this initial trajectory. Nevertheless, these methods are capable of generating smooth trajectories, and although they can be too computationally intensive for dynamic real-time environments, they are generally effective in finding constrained motion plans. The inverse kinematics with task prioritisation, which is based on finding a locally optimal least squares solution, and linear quadratic programming are two of the local optimization methods. Both methods are fast, suitable for real-time applications in dynamic environments and provide smooth solutions. However, since they do not plan further than one step ahead, they tend to get stuck in local minima. Therefore, they are often integrated with a higher-level planner, usually based on sampling, for global path search, while local optimisation takes dynamic environment changes into account.

\add{citations}

% PREDSTAVITEV PRISTOPOV RAČUNANJA ODDALJENOSTI OVIRAM
To be able to generate collision free trajectory we need to have a representation of the environment. One common approach is to represent environment with geomertic primitives such as bounding boxes or spheres. These representations are often combined into hierarhical trees, such as octatree, to make calculations more efficient. Such representations are memory effective and offer simple and computationally undemanding distance calculation and collision detection. These representations are widely used in computer simulations and games, however they can be suboptimal for , where sensor measurements arent always accurate and contain only part of the surrounding space. This can lead to approximation errors, which are additionally probelmatic when we are dealing with irregular shaped objects.

Point-cloud representation captures the environment by directly using a collection of points obtained from sensors like LiDAR, providing a more precise depiction of space than bounding boxes or spheres. This method can more accurately represent the complex and irregular shapes of real-world objects, although it typically requires more memory and computational power to process, which can make it chllenging fore real-time representation. 

Voxel grid representation, on the other hand, divides the space into a regular grid of volumetric elements, or voxels, which can be used to create a more manageable approximation of the environment. While this approach offers a balance between detail and computational efficiency, it can still introduce discretization errors, particularly when modeling objects with smooth surfaces or intricate details. The fidelity of the representation is dependent on the size of the voxels: smaller voxels can capture more detail but require more memory and computation, while larger voxels result in coarser approximations but are more memory-efficient, adaptive voxel grids have been explored, where the voxel size can vary throughout the space to provide higher resolution in regions of interest while conserving resources in less critical areas. Voxel grids can incorporate probabilistic information, such as in occupancy grid maps, where each voxel holds a probability indicating the chance of an obstacle's presence. The occupancy probability of a voxel can be updated dynamically using sensor measurements and Bayesian updating methods. As new sensor data is collected, the probabilities are revised to reflect the increased or decreased likelihood of the presence of an obstacle in the voxel space. 

Additionally, techniques such as ESDF and TSDF can be utilized in conjunction with voxel grids to encode distance information, thereby enhancing the grid's utility for navigation and path planning by providing gradient information for obstacle avoidance and facilitating smoother trajectories

\add{Gradient based voxel field methods. Voxel fields including distance information.}





\clearpage

\section{USING THE TEMPLATE}

Use this sample document as your LaTeX source file to create your document. Save this file as {\bf root.tex}. You have to make sure to use the cls file that came with this distribution. If you use a different style file, you cannot expect to get required margins. Note also that when you are creating your out PDF file, the source file is only part of the equation. {\it Your \TeX\ $\rightarrow$ PDF filter determines the output file size. Even if you make all the specifications to output a letter file in the source - if your filter is set to produce A4, you will only get A4 output. }

It is impossible to account for all possible situation, one would encounter using \TeX. If you are using multiple \TeX\ files you must make sure that the ``MAIN`` source file is called root.tex - this is particularly important if your conference is using PaperPlaza's built in \TeX\ to PDF conversion tool.

\subsection{Headings, etc}

Text heads organize the topics on a relational, hierarchical basis. For example, the paper title is the primary text head because all subsequent material relates and elaborates on this one topic. If there are two or more sub-topics, the next level head (uppercase Roman numerals) should be used and, conversely, if there are not at least two sub-topics, then no subheads should be introduced. Styles named  Heading 1 ,  Heading 2 ,  Heading 3 , and  Heading 4  are prescribed.

\subsection{Figures and Tables}

Positioning Figures and Tables: Place figures and tables at the top and bottom of columns. Avoid placing them in the middle of columns. Large figures and tables may span across both columns. Figure captions should be below the figures; table heads should appear above the tables. Insert figures and tables after they are cited in the text. Use the abbreviation  Fig. 1 , even at the beginning of a sentence.

\begin{table}[h]
\caption{An Example of a Table}
\label{table_example}
\begin{center}
\begin{tabular}{|c||c|}
\hline
One & Two\\
\hline
Three & Four\\
\hline
\end{tabular}
\end{center}
\end{table}


   \begin{figure}[thpb]
      \centering
      \framebox{\parbox{3in}{We suggest that you use a text box to insert a graphic (which is ideally a 300 dpi TIFF or EPS file, with all fonts embedded) because, in an document, this method is somewhat more stable than directly inserting a picture.
}}
      %\includegraphics[scale=1.0]{figurefile}
      \caption{Inductance of oscillation winding on amorphous
       magnetic core versus DC bias magnetic field}
      \label{figurelabel}
   \end{figure}
   

Figure Labels: Use 8 point Times New Roman for Figure labels. Use words rather than symbols or abbreviations when writing Figure axis labels to avoid confusing the reader. As an example, write the quantity  Magnetization , or  Magnetization, M , not just  M . If including units in the label, present them within parentheses. Do not label axes only with units. In the example, write  Magnetization (A/m)  or  Magnetization {A[m(1)]} , not just  A/m . Do not label axes with a ratio of quantities and units. For example, write  Temperature (K) , not  Temperature/K. 

\section{CONCLUSIONS}

A conclusion section is not required. Although a conclusion may review the main points of the paper, do not replicate the abstract as the conclusion. A conclusion might elaborate on the importance of the work or suggest applications and extensions. 

\addtolength{\textheight}{-12cm}   % This command serves to balance the column lengths
                                  % on the last page of the document manually. It shortens
                                  % the textheight of the last page by a suitable amount.
                                  % This command does not take effect until the next page
                                  % so it should come on the page before the last. Make
                                  % sure that you do not shorten the textheight too much.

%%%%%%%%%%%%%%%%%%%%%%%%%%%%%%%%%%%%%%%%%%%%%%%%%%%%%%%%%%%%%%%%%%%%%%%%%%%%%%%%



%%%%%%%%%%%%%%%%%%%%%%%%%%%%%%%%%%%%%%%%%%%%%%%%%%%%%%%%%%%%%%%%%%%%%%%%%%%%%%%%



%%%%%%%%%%%%%%%%%%%%%%%%%%%%%%%%%%%%%%%%%%%%%%%%%%%%%%%%%%%%%%%%%%%%%%%%%%%%%%%%
\section*{APPENDIX}

Appendixes should appear before the acknowledgment.

\section*{ACKNOWLEDGMENT}

The preferred spelling of the word  acknowledgment  in America is without an  e  after the  g . Avoid the stilted expression,  One of us (R. B. G.) thanks . . .   Instead, try  R. B. G. thanks . Put sponsor acknowledgments in the unnumbered footnote on the first page.



%%%%%%%%%%%%%%%%%%%%%%%%%%%%%%%%%%%%%%%%%%%%%%%%%%%%%%%%%%%%%%%%%%%%%%%%%%%%%%%%

References are important to the reader; therefore, each citation must be complete and correct. If at all possible, references should be commonly available publications.



\begin{thebibliography}{99}
	
	\bibitem{IDEASLab2023}
	IDEAS Lab, "Motion and Path Planning," presented at Purdue University, 2023. [Online]. Available: \url{https://ideas.cs.purdue.edu/research/robotics/planning/}. Accessed on: Jan. 9, 2024.
	

\bibitem{c1} G. O. Young,  Synthetic structure of industrial plastics (Book style with paper title and editor),  	in Plastics, 2nd ed. vol. 3, J. Peters, Ed.  New York: McGraw-Hill, 1964, pp. 15 64.
\bibitem{c2} W.-K. Chen, Linear Networks and Systems (Book style).	Belmont, CA: Wadsworth, 1993, pp. 123 135.
\bibitem{c3} H. Poor, An Introduction to Signal Detection and Estimation.   New York: Springer-Verlag, 1985, ch. 4.
\bibitem{c4} B. Smith,  An approach to graphs of linear forms (Unpublished work style),  unpublished.
\bibitem{c5} E. H. Miller,  A note on reflector arrays (Periodical styleÑAccepted for publication),  IEEE Trans. Antennas Propagat., to be publised.
\bibitem{c6} J. Wang,  Fundamentals of erbium-doped fiber amplifiers arrays (Periodical styleÑSubmitted for publication),  IEEE J. Quantum Electron., submitted for publication.
\bibitem{c7} C. J. Kaufman, Rocky Mountain Research Lab., Boulder, CO, private communication, May 1995.
\bibitem{c8} Y. Yorozu, M. Hirano, K. Oka, and Y. Tagawa,  Electron spectroscopy studies on magneto-optical media and plastic substrate interfaces(Translation Journals style),  IEEE Transl. J. Magn.Jpn., vol. 2, Aug. 1987, pp. 740 741 [Dig. 9th Annu. Conf. Magnetics Japan, 1982, p. 301].
\bibitem{c9} M. Young, The Techincal Writers Handbook.  Mill Valley, CA: University Science, 1989.
\bibitem{c10} J. U. Duncombe,  Infrared navigationÑPart I: An assessment of feasibility (Periodical style),  IEEE Trans. Electron Devices, vol. ED-11, pp. 34 39, Jan. 1959.
\bibitem{c11} S. Chen, B. Mulgrew, and P. M. Grant,  A clustering technique for digital communications channel equalization using radial basis function networks,  IEEE Trans. Neural Networks, vol. 4, pp. 570 578, July 1993.
\bibitem{c12} R. W. Lucky,  Automatic equalization for digital communication,  Bell Syst. Tech. J., vol. 44, no. 4, pp. 547 588, Apr. 1965.
\bibitem{c13} S. P. Bingulac,  On the compatibility of adaptive controllers (Published Conference Proceedings style),  in Proc. 4th Annu. Allerton Conf. Circuits and Systems Theory, New York, 1994, pp. 8 16.
\bibitem{c14} G. R. Faulhaber,  Design of service systems with priority reservation,  in Conf. Rec. 1995 IEEE Int. Conf. Communications, pp. 3 8.
\bibitem{c15} W. D. Doyle,  Magnetization reversal in films with biaxial anisotropy,  in 1987 Proc. INTERMAG Conf., pp. 2.2-1 2.2-6.
\bibitem{c16} G. W. Juette and L. E. Zeffanella,  Radio noise currents n short sections on bundle conductors (Presented Conference Paper style),  presented at the IEEE Summer power Meeting, Dallas, TX, June 22 27, 1990, Paper 90 SM 690-0 PWRS.
\bibitem{c17} J. G. Kreifeldt,  An analysis of surface-detected EMG as an amplitude-modulated noise,  presented at the 1989 Int. Conf. Medicine and Biological Engineering, Chicago, IL.
\bibitem{c18} J. Williams,  Narrow-band analyzer (Thesis or Dissertation style),  Ph.D. dissertation, Dept. Elect. Eng., Harvard Univ., Cambridge, MA, 1993. 
\bibitem{c19} N. Kawasaki,  Parametric study of thermal and chemical nonequilibrium nozzle flow,  M.S. thesis, Dept. Electron. Eng., Osaka Univ., Osaka, Japan, 1993.
\bibitem{c20} J. P. Wilkinson,  Nonlinear resonant circuit devices (Patent style),  U.S. Patent 3 624 12, July 16, 1990. 






\end{thebibliography}




\end{document}
