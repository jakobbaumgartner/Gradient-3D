%%%%%%%%%%%%%%%%%%%%%%%%%%%%%%%%%%%%%%%%%%%%%%%%%%%%%%%%%%%%%%%%%%%%%%%%%%%%%%%%
%2345678901234567890123456789012345678901234567890123456789012345678901234567890
%        1         2         3         4         5         6         7         8

\documentclass[letterpaper, 10 pt, conference]{ieeeconf}  % Comment this line out if you need a4paper



%\documentclass[a4paper, 10pt, conference]{ieeeconf}      % Use this line for a4 paper

\IEEEoverridecommandlockouts                              % This command is only needed if 
                                                          % you want to use the \thanks command

\overrideIEEEmargins                                      % Needed to meet printer requirements.

%In case you encounter the following error:
%Error 1010 The PDF file may be corrupt (unable to open PDF file) OR
%Error 1000 An error occurred while parsing a contents stream. Unable to analyze the PDF file.
%This is a known problem with pdfLaTeX conversion filter. The file cannot be opened with acrobat reader
%Please use one of the alternatives below to circumvent this error by uncommenting one or the other
%\pdfobjcompresslevel=0
%\pdfminorversion=4

% See the \addtolength command later in the file to balance the column lengths
% on the last page of the document

% The following packages can be found on http:\\www.ctan.org
%\usepackage{graphics} % for pdf, bitmapped graphics files
%\usepackage{epsfig} % for postscript graphics files
%\usepackage{mathptmx} % assumes new font selection scheme installed
%\usepackage{times} % assumes new font selection scheme installed
%\usepackage{amsmath} % assumes amsmath package installed
%\usepackage{amssymb}  % assumes amsmath package installed
\usepackage{hyperref}


 \usepackage{easyReview}
 


\title{\LARGE \bf
Preparation of Papers for IEEE Sponsored Conferences \& Symposia*
}


\author{Albert Author$^{1}$ and Bernard D. Researcher$^{2}$% <-this % stops a space
\thanks{*This work was not supported by any organization}% <-this % stops a space
\thanks{$^{1}$Albert Author is with Faculty of Electrical Engineering, Mathematics and Computer Science,
        University of Twente, 7500 AE Enschede, The Netherlands
        {\tt\small albert.author@papercept.net}}%
\thanks{$^{2}$Bernard D. Researcheris with the Department of Electrical Engineering, Wright State University,
        Dayton, OH 45435, USA
        {\tt\small b.d.researcher@ieee.org}}%
}


\begin{document}



\maketitle
\thispagestyle{empty}
\pagestyle{empty}


%%%%%%%%%%%%%%%%%%%%%%%%%%%%%%%%%%%%%%%%%%%%%%%%%%%%%%%%%%%%%%%%%%%%%%%%%%%%%%%%
\begin{abstract}

This electronic document is a Olive template. The various components of your paper [title, text, heads, etc.] are already defined on the style sheet, as illustrated by the portions given in this document.

\end{abstract}


%%%%%%%%%%%%%%%%%%%%%%%%%%%%%%%%%%%%%%%%%%%%%%%%%%%%%%%%%%%%%%%%%%%%%%%%%%%%%%%%
\section{INTRODUCTION}

% PREDSTAVITEV REDUNDANCE

Kinematic redundancy enables a kinematically redundant manipulator to follow a predefined task space trajectory using the endeffector (EE), while simultaneously, executing an additional task with the remaining movement capacity without impacting the trajectory adherence. This is possible because the robot's degrees of freedom (DOF) go beyond what is required to perform the primary task. Consequently, the robot can adopt different joint configurations optimised according to the secondary task while performing the primary task. Common secondary tasks include avoiding singularities, optimising the manipulability measure, minimising joint torques and avoiding obstacles in the operating space.

\add{reference za posamezne pristope, dela \\}

% PREDSTAVITEV PRISTOPOV PLANIRANJA POTI V ROBOTIKI
\add{Sampling based, Kinematic Optimizazion (LS, LQP), GCS, Learning-based, DMP(Hoffman), Trajectory Optimization.
\\}

Motion planning is a fundamental problem in robotics \cite{IDEASLab2023}. It consists of finding a sequence of joint configurations for a robot so that the robot can move along this path from its initial configuration to the goal configuration without colliding with static obstacles or other robots in the environment. In addition to collision avoidance, motion planning for manipulators can optionally take into account various constraints, such as position, velocity, acceleration or jerk constraints for the joint angle or end effector, precision of the end effector with respect to position and orientation, stability of the manipulator, avoidance of singularities, or any number of other criteria.



%
% PREDSTAVITEV PRISTOPOV RAČUNANJA ODDALJENOSTI OVIRAM

\add{
	Nato želimo omeniti, da so ti ljudje se problemu približali iz vidika izračuna oddaljenosti objektov. Omeniti je potrebno pristope z "bouding boxes AABB itd" in potem opisati "voxel grids". Končno je potrebno omeniti pristope z vektorskimi polji in nato pristope izračuna oddaljenosti.}







\clearpage

\section{USING THE TEMPLATE}

Use this sample document as your LaTeX source file to create your document. Save this file as {\bf root.tex}. You have to make sure to use the cls file that came with this distribution. If you use a different style file, you cannot expect to get required margins. Note also that when you are creating your out PDF file, the source file is only part of the equation. {\it Your \TeX\ $\rightarrow$ PDF filter determines the output file size. Even if you make all the specifications to output a letter file in the source - if your filter is set to produce A4, you will only get A4 output. }

It is impossible to account for all possible situation, one would encounter using \TeX. If you are using multiple \TeX\ files you must make sure that the ``MAIN`` source file is called root.tex - this is particularly important if your conference is using PaperPlaza's built in \TeX\ to PDF conversion tool.

\subsection{Headings, etc}

Text heads organize the topics on a relational, hierarchical basis. For example, the paper title is the primary text head because all subsequent material relates and elaborates on this one topic. If there are two or more sub-topics, the next level head (uppercase Roman numerals) should be used and, conversely, if there are not at least two sub-topics, then no subheads should be introduced. Styles named  Heading 1 ,  Heading 2 ,  Heading 3 , and  Heading 4  are prescribed.

\subsection{Figures and Tables}

Positioning Figures and Tables: Place figures and tables at the top and bottom of columns. Avoid placing them in the middle of columns. Large figures and tables may span across both columns. Figure captions should be below the figures; table heads should appear above the tables. Insert figures and tables after they are cited in the text. Use the abbreviation  Fig. 1 , even at the beginning of a sentence.

\begin{table}[h]
\caption{An Example of a Table}
\label{table_example}
\begin{center}
\begin{tabular}{|c||c|}
\hline
One & Two\\
\hline
Three & Four\\
\hline
\end{tabular}
\end{center}
\end{table}


   \begin{figure}[thpb]
      \centering
      \framebox{\parbox{3in}{We suggest that you use a text box to insert a graphic (which is ideally a 300 dpi TIFF or EPS file, with all fonts embedded) because, in an document, this method is somewhat more stable than directly inserting a picture.
}}
      %\includegraphics[scale=1.0]{figurefile}
      \caption{Inductance of oscillation winding on amorphous
       magnetic core versus DC bias magnetic field}
      \label{figurelabel}
   \end{figure}
   

Figure Labels: Use 8 point Times New Roman for Figure labels. Use words rather than symbols or abbreviations when writing Figure axis labels to avoid confusing the reader. As an example, write the quantity  Magnetization , or  Magnetization, M , not just  M . If including units in the label, present them within parentheses. Do not label axes only with units. In the example, write  Magnetization (A/m)  or  Magnetization {A[m(1)]} , not just  A/m . Do not label axes with a ratio of quantities and units. For example, write  Temperature (K) , not  Temperature/K. 

\section{CONCLUSIONS}

A conclusion section is not required. Although a conclusion may review the main points of the paper, do not replicate the abstract as the conclusion. A conclusion might elaborate on the importance of the work or suggest applications and extensions. 

\addtolength{\textheight}{-12cm}   % This command serves to balance the column lengths
                                  % on the last page of the document manually. It shortens
                                  % the textheight of the last page by a suitable amount.
                                  % This command does not take effect until the next page
                                  % so it should come on the page before the last. Make
                                  % sure that you do not shorten the textheight too much.

%%%%%%%%%%%%%%%%%%%%%%%%%%%%%%%%%%%%%%%%%%%%%%%%%%%%%%%%%%%%%%%%%%%%%%%%%%%%%%%%



%%%%%%%%%%%%%%%%%%%%%%%%%%%%%%%%%%%%%%%%%%%%%%%%%%%%%%%%%%%%%%%%%%%%%%%%%%%%%%%%



%%%%%%%%%%%%%%%%%%%%%%%%%%%%%%%%%%%%%%%%%%%%%%%%%%%%%%%%%%%%%%%%%%%%%%%%%%%%%%%%
\section*{APPENDIX}

Appendixes should appear before the acknowledgment.

\section*{ACKNOWLEDGMENT}

The preferred spelling of the word  acknowledgment  in America is without an  e  after the  g . Avoid the stilted expression,  One of us (R. B. G.) thanks . . .   Instead, try  R. B. G. thanks . Put sponsor acknowledgments in the unnumbered footnote on the first page.



%%%%%%%%%%%%%%%%%%%%%%%%%%%%%%%%%%%%%%%%%%%%%%%%%%%%%%%%%%%%%%%%%%%%%%%%%%%%%%%%

References are important to the reader; therefore, each citation must be complete and correct. If at all possible, references should be commonly available publications.



\begin{thebibliography}{99}
	
	\bibitem{IDEASLab2023}
	IDEAS Lab, "Motion and Path Planning," presented at Purdue University, 2023. [Online]. Available: \url{https://ideas.cs.purdue.edu/research/robotics/planning/}. Accessed on: Jan. 9, 2024.
	

\bibitem{c1} G. O. Young,  Synthetic structure of industrial plastics (Book style with paper title and editor),  	in Plastics, 2nd ed. vol. 3, J. Peters, Ed.  New York: McGraw-Hill, 1964, pp. 15 64.
\bibitem{c2} W.-K. Chen, Linear Networks and Systems (Book style).	Belmont, CA: Wadsworth, 1993, pp. 123 135.
\bibitem{c3} H. Poor, An Introduction to Signal Detection and Estimation.   New York: Springer-Verlag, 1985, ch. 4.
\bibitem{c4} B. Smith,  An approach to graphs of linear forms (Unpublished work style),  unpublished.
\bibitem{c5} E. H. Miller,  A note on reflector arrays (Periodical styleÑAccepted for publication),  IEEE Trans. Antennas Propagat., to be publised.
\bibitem{c6} J. Wang,  Fundamentals of erbium-doped fiber amplifiers arrays (Periodical styleÑSubmitted for publication),  IEEE J. Quantum Electron., submitted for publication.
\bibitem{c7} C. J. Kaufman, Rocky Mountain Research Lab., Boulder, CO, private communication, May 1995.
\bibitem{c8} Y. Yorozu, M. Hirano, K. Oka, and Y. Tagawa,  Electron spectroscopy studies on magneto-optical media and plastic substrate interfaces(Translation Journals style),  IEEE Transl. J. Magn.Jpn., vol. 2, Aug. 1987, pp. 740 741 [Dig. 9th Annu. Conf. Magnetics Japan, 1982, p. 301].
\bibitem{c9} M. Young, The Techincal Writers Handbook.  Mill Valley, CA: University Science, 1989.
\bibitem{c10} J. U. Duncombe,  Infrared navigationÑPart I: An assessment of feasibility (Periodical style),  IEEE Trans. Electron Devices, vol. ED-11, pp. 34 39, Jan. 1959.
\bibitem{c11} S. Chen, B. Mulgrew, and P. M. Grant,  A clustering technique for digital communications channel equalization using radial basis function networks,  IEEE Trans. Neural Networks, vol. 4, pp. 570 578, July 1993.
\bibitem{c12} R. W. Lucky,  Automatic equalization for digital communication,  Bell Syst. Tech. J., vol. 44, no. 4, pp. 547 588, Apr. 1965.
\bibitem{c13} S. P. Bingulac,  On the compatibility of adaptive controllers (Published Conference Proceedings style),  in Proc. 4th Annu. Allerton Conf. Circuits and Systems Theory, New York, 1994, pp. 8 16.
\bibitem{c14} G. R. Faulhaber,  Design of service systems with priority reservation,  in Conf. Rec. 1995 IEEE Int. Conf. Communications, pp. 3 8.
\bibitem{c15} W. D. Doyle,  Magnetization reversal in films with biaxial anisotropy,  in 1987 Proc. INTERMAG Conf., pp. 2.2-1 2.2-6.
\bibitem{c16} G. W. Juette and L. E. Zeffanella,  Radio noise currents n short sections on bundle conductors (Presented Conference Paper style),  presented at the IEEE Summer power Meeting, Dallas, TX, June 22 27, 1990, Paper 90 SM 690-0 PWRS.
\bibitem{c17} J. G. Kreifeldt,  An analysis of surface-detected EMG as an amplitude-modulated noise,  presented at the 1989 Int. Conf. Medicine and Biological Engineering, Chicago, IL.
\bibitem{c18} J. Williams,  Narrow-band analyzer (Thesis or Dissertation style),  Ph.D. dissertation, Dept. Elect. Eng., Harvard Univ., Cambridge, MA, 1993. 
\bibitem{c19} N. Kawasaki,  Parametric study of thermal and chemical nonequilibrium nozzle flow,  M.S. thesis, Dept. Electron. Eng., Osaka Univ., Osaka, Japan, 1993.
\bibitem{c20} J. P. Wilkinson,  Nonlinear resonant circuit devices (Patent style),  U.S. Patent 3 624 12, July 16, 1990. 






\end{thebibliography}




\end{document}
